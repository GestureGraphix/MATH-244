\documentclass{report}

\input{preamble}
\input{macros}
\input{letterfonts}

\title{\Huge{Math 244}}
\author{\huge{PSET 2}}
\date{Feb 3 2025}

\begin{document}

\maketitle
\newpage% or \cleardoublepage
% \pdfbookmark[<level>]{<title>}{<dest>}
\pdfbookmark[section]{\contentsname}{toc}
\tableofcontents
\pagebreak

\section*{Section 1.5 Problem 5}
\addcontentsline{toc}{section}{Problem 1}

\qs{}{
  Prove the associativity of composing relations: If $R$, $S$, and $T$ are relations such that 
  $( R \circ S) \circ T$ is well defined, then $R \circ (S \circ T)$ is well defined and equal to $( R \circ S) \circ T$.
}

\begin{RemarkWithLily}{For prob 1}
  WWe know $(w,z)$ lies in $(R \circ S) \circ T$ \textit{iff} there is some $y$ such that 
  \begin{itemize}
    \item [1] $(w,y) \in R \circ S $ and 
    \item [2] $(y,z) \in T$ 
  \end{itemize}

  $(w, y) \in R \circ S$ means: 

  \[ (w,x) \in R \quad \text{and} \quad (x,y) \in S\]
  
  So we have $(x,y) \in S$ and $(y,z) \in T $, which implies $(y, z) \in S \circ T$ So, $(w,x) \in R$ and 
  $(y,z) \in S \circ T$ means that $(w,z) \in R \circ (S \circ T)$ 

  $(R \circ S) \circ T \subseteq R \circ (S \circ T)$ 



\end{RemarkWithLily}

\section*{Section 1.6 Problem 3}
\addcontentsline{toc}{section}{Problem 2}

\qs{}{
  Prove that a relation $R$ is transitive if and only if $R \circ R \subseteq R$.
}

\begin{keyideaWithLotus}
  We need to show 
  \begin{itemize}
    \item [$\Rightarrow$] If $R$ is transitive, then, $R \circ R \subseteq R$
    \item [$\Leftarrow$] If $R \circ R \subseteq R$, then $R$ is transitive.   
  \end{itemize}

\end{keyideaWithLotus}

\begin{RemarkWithLily}{$\Rightarrow$}
  WWe assume that $R$ is transitive. \\
  By def or relational composition we know that for a pair $(a,c) \in R \circ R$ there exists some $b$ $s.t.$ 

  \[ (a,b) \in R \quad \text{and} \quad (b,c) \in R \]
  
  
\end{RemarkWithLily}

\section*{Section 1.6 Problem 6}
\addcontentsline{toc}{section}{Problem 3}

\qs{}{
  Describe all relations on a set $X$ that are equivalences and orderings at the same time. 
}

\section*{Section 2.1 Problem 4}
\addcontentsline{toc}{section}{Problem 4}


\qs{}{
  Let $(X, \preceq)$, $(Y, \preceq)$ be ordered sets. We say that they are \textit{isomorphic} if there exists a bijection $f: X \to Y$ such that for every $x, y \in X$, we have 
  $x \leq y$ if and only if $f(x) \preceq f(y)$. 

  \begin{itemize}
    \item [a)] Draw Hasse diagrams for all non-isomorphic ordered sets with 3 elements posets.
    \item [b)] Prove that any two $n$-elements linearly ordered sets are isomorphic.
  \end{itemize}
}


\section*{Section 2.2 Problem 3}
\addcontentsline{toc}{section}{Problem 5}

\qs{}{
  \begin{itemize}
    \item [a)] Consider the set $\{ 1, 2, \ldots n \}$ ordered by the divisibility relation $|$. What is the maximum possible number of elements of a set $X \subseteq \{1, 2, \ldots n \}$ that is ordered linearly by the relation $|$ 
    \item [b)] Solve the same question for the set $2^{\{1, 2, \ldots n\}}$ ordered by the inclusion relation $\subseteq$.
  \end{itemize}
}



\end{document}