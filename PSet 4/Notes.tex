\documentclass{report}

\input{preamble}
\input{macros}
\input{letterfonts}

\title{\Huge{Math 244}}
\author{\huge{PSET 4}}
\date{Feb 21 2025}

\begin{document}

\maketitle
\newpage% or \cleardoublepage
% \pdfbookmark[<level>]{<title>}{<dest>}
\pdfbookmark[section]{\contentsname}{toc}
\tableofcontents
\pagebreak


\section*{3.2 Prob 1}
\addcontentsline{toc}{section}{3.2 Prob 1}

\qs{}{
  How many permutations of \(\{1,2,\dots,n\}\) have a single cycle?
}

\begin{proofWithHibiscus}
  There are \( n! \) ways to arrange \( n \) distinct elements in a sequence. Any sequence
  \[
  (a_1, a_2, \dots, a_n)
  \]
  can be thought of as a candidate for an \( n \)-cycle when interpreted in cyclic notation as
  \[
  (a_1, a_2, \dots, a_n).
  \]
  
  But we can't count cycles where the order is simply rotated cyclically
  \[
  (a_1, a_2, \dots, a_n)
  \]
  is the same as the cycle
  \[
  (a_2, a_3, \dots, a_n, a_1),
  \]
  there are exactly \( n \) different ways to write the same cycle by cyclically rotating the elements. \\
  
  If we count every sequence as a distinct cycle, we overcount each \( n \)-cycle \( n \) times. \\
    
  To deal with the overcounting we divide the total number of sequences by \( n \). 
  \[
  \frac{n!}{n} = (n-1)!.
  \]
  
\end{proofWithHibiscus}


\section*{3.2 Prob 2}
\addcontentsline{toc}{section}{3.2 Prob 2}

\qs{}{
  For a permutation \( p: X \to X \), let \( p^k \) denote the permutation arising by a \( k \)-fold 
  composition of \( p \), i.e., \( p^1 = p \) and \( p^k = p \circ p^{k-1} \). Define a relation 
  \( \approx \) on the set \( X \) as follows: \( i \approx j \) if and only if there exists a 
  \( k \geq 1 \) such that \( p^k(i) = j \). Prove that \( \approx \) is an equivalence relation 
  on \( X \), and that its classes are the cycles of \( p \).
}

\begin{proofWithHibiscus}
  Want to show that \( \approx \) satisfies the three properties of an equivalence relation: reflexivity, symmetry, and transitivity.

  \medskip
    
  For any \( i \in X \), since \( p \) is a permutation meaning its bijective, \( i \) must belong to some cycle. So, there exists a smallest positive integer \( k \geq 1 \) such that
  \[ p^k(i) = i. \]
  By the definition of \( \approx \), we have \( i \approx i \). So, \( \approx \) is reflexive.
  
  \medskip

  Suppose \( i \approx j \). Then there exists an integer \( k \geq 1 \) such that
  \[ p^k(i) = j. \]
  Since \( p \) is a permutation, it has an inverse \( p^{-1} \). Applying \( p^{-k} \) to both sides, we get
  \[ i = p^{-k}(j). \]
  Since this is also a permutation, there exists some positive integer \( m \) such that \( p^m(j) = i \). Thus, \( j \approx i \), showing that \( \approx \) is symmetric.
  
  \medskip 

  Suppose \( i \approx j \) and \( j \approx k \). Then there exist positive integers \( m \) and \( n \) such that
  \[ p^m(i) = j \quad \text{and} \quad p^n(j) = k. \]
  Now, consider the composition:
  \[ p^{m+n}(i) = p^n(p^m(i)) = p^n(j) = k. \]
  Since \( m+n \geq 1 \), it follows that \( i \approx k \). So, \( \approx \) is transitive.
  
  \bigskip   

  In cycle notation, the cycle containing an element \( i \) is defined as:
  \[\{ i,\, p(i),\, p^2(i),\, \ldots,\, p^{k-1}(i) \}, \]
  where \( k \) is the smallest positive integer such that \( p^k(i) = i \). 

  \bigskip
  
  By our definition of \( \approx \), an element \( j \in X \) is related to \( i \) if and only if there exists some \( k \geq 1 \) such that \( p^k(i) = j \). 
  This is exactly the description of the cycle of \( i \). So, the equivalence 
  class of \( i \) under \( \approx \) is precisely the set of all elements that 
  can be reached from \( i \) by some power of \( p \), which corresponds to the 
  cycle of \( i \).
    
\end{proofWithHibiscus}

\section*{3.3 Prob 7}
\addcontentsline{toc}{section}{3.3 Prob 7}

\qs{}{
  How many functions \( f: \{1,2, \dots, n\} \to \{1,2, \dots, n\} \) are there 
  that are \textit{monotonic}; that is, for \( i < j \) we have \( f(i) \leq f(j) \)? 
  \textit{The textbook has a hint to this problem in the back.}
}

\begin{proofWithHibiscus}
  Define a new sequence \( g \) by
  \[
  g(i) = f(i) + i - 1, \quad \text{for } i=1,2,\dots,n.
  \]
  Since \( f(1)\le f(2)\le \cdots \le f(n) \), it follows that
  \[
  g(1) = f(1) \le f(2) + 1 = g(2) \le \cdots \le f(n) + n - 1 = g(n).
  \]
  Because we are adding the strictly increasing sequence \( 0,1,2,\dots,n-1 \) to \( f(i) \), the sequence \( g(1), g(2), \dots, g(n) \) becomes strictly increasing:
  \[
  g(1) < g(2) < \cdots < g(n).
  \]
  
  \bigskip 

  Since \( f(i) \in \{1,2,\dots,n\} \), we obtain:
  \begin{itemize}
      \item For \( i=1 \): \( g(1) = f(1) \ge 1 \).
      \item For \( i=n \): \( g(n) = f(n) + n - 1 \le n + n - 1 = 2n - 1 \).
  \end{itemize}
  
  Thus, each \( g(i) \) is a distinct element from the set \( \{1, 2, \dots, 2n-1\} \).
  
  \bigskip
  
  A strictly increasing sequence \( g(1) < g(2) < \cdots < g(n) \) chosen from \( \{1,2,\dots,2n-1\} \) is equivalent to choosing an \( n \)-element subset from a set of \( 2n-1 \) elements. There are
  \[
  \binom{2n-1}{n}
  \]
  ways to select such a subset.
    
  Since the transformation \( f(i) \mapsto g(i) = f(i) + i - 1 \) is a bijection between the set of monotonic functions \( f \) and the set of strictly increasing sequences \( g \), the number of monotonic functions is given by:
  \[
  \boxed{\binom{2n-1}{n}}.
  \]
\end{proofWithHibiscus}

\section*{3.3 Prob 21}
\addcontentsline{toc}{section}{3.3 Prob 21}

\qs{}{
  (optional bonus problem) Draw a triangle \( ABC \). Draw \( n \) points lying on the side \( AB \) (but different from \( A \) and \( B \)) and connect them all by segments to the vertex \( C \). Similarly, draw \( n \) points on the side \( AC \) and connect them to \( B \).
    
  \begin{enumerate}
    \item How many intersections of the drawn segments are there? Into how many regions is the triangle \( ABC \) partitioned by the drawn segments?

    \item Draw \( n \) points on the side \( BC \) and connect them to \( A \). Assume that no 3 of the drawn segments intersect at a single point. How many intersections are there now?

    \item How many regions are there in the situation of (b)?
  \end{enumerate}
}

\section*{3.7 Prob 3}
\addcontentsline{toc}{section}{3.7 Prob 3}

\qs{}{
  (Sieve of Eratosthenes) How many numbers are left in the set 
  \(\{1,2,\dots,1000\}\) after all multiples of 2, 3, 5, and 7
  are crossed out?
}

\begin{RemarkWithLily}{For Prob 4}
  DDefine:

  \begin{itemize}
      \item \(A_2\) as the set of multiples of 2,
      \item \(A_3\) as the set of multiples of 3,
      \item \(A_5\) as the set of multiples of 5,
      \item \(A_7\) as the set of multiples of 7.
  \end{itemize}
  
  We want the size of the complement of \(A_2 \cup A_3 \cup A_5 \cup A_7\)
  
  We cound the multiples of \(2,3,5,7 \) in the range of [1,1000].

  \[
  \begin{aligned}
  |A_2| &= \left\lfloor \frac{1000}{2} \right\rfloor = 500, \\
  |A_3| &= \left\lfloor \frac{1000}{3} \right\rfloor = 333, \\
  |A_5| &= \left\lfloor \frac{1000}{5} \right\rfloor = 200, \\
  |A_7| &= \left\lfloor \frac{1000}{7} \right\rfloor = 142.
  \end{aligned}
  \]
  
  count of multiples for pairs using least common multiple:
  \[
  \begin{aligned}
  |A_2 \cap A_3| &= \left\lfloor \frac{1000}{6} \right\rfloor = 166, \\
  |A_2 \cap A_5| &= \left\lfloor \frac{1000}{10} \right\rfloor = 100, \\
  |A_2 \cap A_7| &= \left\lfloor \frac{1000}{14} \right\rfloor = 71, \\
  |A_3 \cap A_5| &= \left\lfloor \frac{1000}{15} \right\rfloor = 66, \\
  |A_3 \cap A_7| &= \left\lfloor \frac{1000}{21} \right\rfloor = 47, \\
  |A_5 \cap A_7| &= \left\lfloor \frac{1000}{35} \right\rfloor = 28.
  \end{aligned}
  \]
  
  Count of multiples for triples
  \[
  \begin{aligned}
  |A_2 \cap A_3 \cap A_5| &= \left\lfloor \frac{1000}{30} \right\rfloor = 33, \\
  |A_2 \cap A_3 \cap A_7| &= \left\lfloor \frac{1000}{42} \right\rfloor = 23, \\
  |A_2 \cap A_5 \cap A_7| &= \left\lfloor \frac{1000}{70} \right\rfloor = 14, \\
  |A_3 \cap A_5 \cap A_7| &= \left\lfloor \frac{1000}{105} \right\rfloor = 9.
  \end{aligned}
  \]
  
  Count of multiples which all four share 
  \[
  |A_2 \cap A_3 \cap A_5 \cap A_7| = \left\lfloor \frac{1000}{210} \right\rfloor = 4.
  \]
  
  Apply the Inclusion-Exclusion Principle
  
  The count of numbers divisible by at least one of \(2,3,5,7\) is:
  
  \[
  \begin{aligned}
  |A_2 \cup A_3 \cup A_5 \cup A_7| &= |A_2|+|A_3|+|A_5|+|A_7| \\
  &\quad - \Bigl(|A_2 \cap A_3| + |A_2 \cap A_5| + |A_2 \cap A_7| + |A_3 \cap A_5| + |A_3 \cap A_7| + |A_5 \cap A_7|\Bigr)\\
  &\quad + \Bigl(|A_2 \cap A_3 \cap A_5| + |A_2 \cap A_3 \cap A_7| + |A_2 \cap A_5 \cap A_7| + |A_3 \cap A_5 \cap A_7|\Bigr)\\
  &\quad - |A_2 \cap A_3 \cap A_5 \cap A_7|.
  \end{aligned}
  \]
  
  Plugging in values:
  \[
  \begin{aligned}
  |A_2 \cup A_3 \cup A_5 \cup A_7| &= 500 + 333 + 200 + 142 \\
  &\quad - (166 + 100 + 71 + 66 + 47 + 28) \\
  &\quad + (33 + 23 + 14 + 9) \\
  &\quad - 4.
  \end{aligned}
  \]
  
  Computing 
  \[
  \begin{aligned}
  500+333+200+142 &= 1175, \\
  166+100+71+66+47+28 &= 478, \\
  33+23+14+9 &= 79.
  \end{aligned}
  \]
  
  So,
  \[
  |A_2 \cup A_3 \cup A_5 \cup A_7| = 1175 - 478 + 79 - 4.
  \]
  
  Evaluating:
  \[
  \begin{aligned}
  1175 - 478 &= 697, \\
  697 + 79 &= 776, \\
  776 - 4 &= 772.
  \end{aligned}
  \]
  
  The count of numbers \textbf{not} divisible by 2, 3, 5, or 7 is:
  \[
  1000 - 772 = 228.
  \]

\end{RemarkWithLily}

\section*{3.8 Prob 4}
\addcontentsline{toc}{section}{3.8 Prob 4}

\qs{}{
  Prove the equation
    \[
    D(n) = n! - nD(n-1) - \binom{n}{2} D(n-2) - \dots - \binom{n}{n-1} D(1) - 1.
    \]
}
\end{document}